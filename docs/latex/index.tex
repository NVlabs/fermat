

\begin{DoxyParagraph}{}
~\newline
 \href{https://github.com/NVlabs/fermat}{\tt Fermat}, is a high performance research oriented physically based rendering system, trying to produce beautiful pictures following the mathematician\textquotesingle{}s principle of least time. ~\newline
 It is a C\+U\+DA physically based research renderer designed and developed by Jacopo Pantaleoni at N\+V\+I\+D\+IA. Its purpose is mostly educational\+: it is primarily intended to teach how to write rendering algorithms, ranging from simple forward path tracing, to bidirectional path tracing, to Metropolis light transport with many of its variants, and do so on massively parallel hardware. ~\newline
 The choice of C\+U\+DA C++ has been made for various reasons\+: the first was to allow the highest level of expression and flexibility in terms of programmability (for example, with template based meta-\/programming); the second was perhaps historical\+: when Fermat\textquotesingle{}s development was started, other ray tracing platforms like Microsoft D\+XR did not exist yet. The ray tracing core employed by Fermat is OptiX -\/ though it is possible that future versions might also offer a D\+XR or a Vulkan backend. ~\newline
 Fermat is built on top of another library co-\/developed for this project\+: \hyperlink{cugar_page}{C\+U\+G\+AR} -\/ C\+U\+DA Graphics AcceleratoR. This is a template library of low-\/level graphics tools, including algorithms for B\+VH, Kd-\/tree and octree construction, sphericals harmonics, sampling, and so on and so on. While packaged together, C\+U\+G\+AR can be thought of as a separate educational project by itself. More information can be found in the relevant Doxygen documentation.
\end{DoxyParagraph}
\hypertarget{index_FermatIntroductionSection}{}\section{Introduction}\label{index_FermatIntroductionSection}
\begin{DoxyParagraph}{}
At the highest level, Fermat is basically a collection of separate renderers. All renderers use Optix Prime as a ray tracing backend, and implement recursive ray / path tracing as an iterative process, in which rays representing new path segments are spawned in {\itshape wavefronts}. This means each renderer is pretty much written like a pipeline of interleaved {\itshape tracing} and {\itshape shading} stages communicating through global memory queues. ~\newline
 Currently, the list of available renderers include\+: ~\newline

\begin{DoxyItemize}
\item \hyperlink{struct_path_tracer}{Path\+Tracer} ({\itshape PT})\+: a simple forward path tracer with next-\/event estimation
\item \hyperlink{struct_p_s_f_p_t}{P\+S\+F\+PT}\+: a path space filtering path tracer, following ideas first developed by Sascha Fricke, Nikolaus Binder and Alex Keller\+: \href{https://dl.acm.org/citation.cfm?id=3214806}{\tt Fast path space filtering by jittered spatial hashing}, Binder et al, A\+CM S\+I\+G\+G\+R\+A\+PH 2018 Talks.
\item \hyperlink{struct_b_p_t}{B\+PT}\+: a bidirectional path tracer
\item \hyperlink{struct_m_l_t}{M\+LT}\+: a path space Metropolis sampler, inspired by the original formulation from Eric Veach\+: \href{https://graphics.stanford.edu/papers/metro/metro.pdf}{\tt Metropolis light transport}, Veach and Guibas, Proceedings of the 24th annual conference on Computer graphics and interactive techniques, 1997 -\/ Pages 65-\/76
\item \hyperlink{struct_p_s_s_m_l_t}{P\+S\+S\+M\+LT}\+: a primary sample space Metropolis sampler, inspired by the seminal paper\+: \href{https://onlinelibrary.wiley.com/doi/pdf/10.1111/1467-8659.t01-1-00703}{\tt A Simple and Robust Mutation Strategy for the Metropolis Light Transport Algorithm}, Kelemen et al, Computer Graphics Forum, Volume21, Issue3, September 2002 -\/ Pages 531-\/540
\item \hyperlink{struct_c_m_l_t}{C\+M\+LT}\+: a Charted Metropolis light transport sampler, as described in\+: \href{https://arxiv.org/abs/1612.05395}{\tt Charted Metropolis Light Transport}, Jacopo Pantaleoni, A\+CM Transactions on Graphics, Volume 36 Issue 4, July 2017.
\item \hyperlink{struct_r_p_t}{R\+PT}\+: a reuse-\/based path tracer inspired by the paper\+: \href{https://www.researchgate.net/publication/220853005_Accelerating_Path_Tracing_by_Re-Using_Paths}{\tt Accelerating path tracing by re-\/using path}, by Bekaert et al E\+G\+RW \textquotesingle{}02 Proceedings of the 13th Eurographics workshop on Rendering Pages 125-\/134 
\end{DoxyItemize}
\end{DoxyParagraph}
\begin{DoxyParagraph}{}
All renderers are implemented on top of a single virtual interface, namely the \hyperlink{struct_renderer_interface}{Renderer\+Interface}. Among the various methods this class may implement (e.\+g. for self initialization, registering auxiliary frame-\/buffer channels, handling mouse and keyboard events, etc), it must implement a single method\+: 
\begin{DoxyCode}
\textcolor{keywordtype}{void} \hyperlink{struct_renderer_interface_aa64254dd44c94929b05092dc8d74f29d}{RendererInterface::render}(\textcolor{keyword}{const} uint32 instance, 
      \hyperlink{struct_rendering_context}{RenderingContext}& rendering\_context);
\end{DoxyCode}
 which takes a frame {\itshape instance} number, useful for progressive rendering, and a reference to a \hyperlink{struct_rendering_context}{Rendering\+Context} object\+: the latter encodes all the information belonging to the current rendering context, including the scene mesh, the output frame-\/buffer, and so on and so on. ~\newline
 Finally, all unidirectional renderers are based on a common, template-\/based library for path tracing, described in the \hyperlink{group___p_t_lib}{P\+T\+Lib} module, while all bidirectional renderers are based on a similar template-\/based library for bidirectional path tracing, described in the \hyperlink{group___b_p_t_lib}{B\+P\+T\+Lib} module.
\end{DoxyParagraph}
\hypertarget{index_FermatCompilationSection}{}\section{Compiling Fermat}\label{index_FermatCompilationSection}
\begin{DoxyParagraph}{}
Assuming you have Windows 10.\+0, Visual Studio 2015, C\+U\+DA 10.\+0 and OptiX 6 installed (driver version 418.\+81 and above), compiling Fermat should be straightforward. It will be sufficient to open the VS solution located in {\itshape ..sln}, and press Build -\/$>$ Build Solution. Note that there are three different projects\+:
\begin{DoxyItemize}
\item {\bfseries fermatdll}\+: Fermat\textquotesingle{}s core library
\item {\bfseries fermat}\+: the main executable, linking fermat.\+dll
\item {\bfseries hellopt}\+: an example plugin renderer
\end{DoxyItemize}
\end{DoxyParagraph}
\hypertarget{index_FermatDocumentationSection}{}\section{Documentation}\label{index_FermatDocumentationSection}
\hypertarget{index_FermatDelivingIntoThePitSection}{}\subsection{Delving into the Pit}\label{index_FermatDelivingIntoThePitSection}
\begin{DoxyParagraph}{}
If you truly want to know more about Fermat\textquotesingle{}s internals you could certainly just go and look at the code, or skim through the \hyperlink{_modules_page}{Module List}, though admittedly that\textquotesingle{}s something that even the author would hardly ever do with somebody else\textquotesingle{}s codebase. Hopefully, a much better alternative is to follow this link that will guide you through a step by step explanation which, at least in principle, pretends to be written in a more or less logical order\+: 
\end{DoxyParagraph}
\begin{DoxyParagraph}{}
\hyperlink{_overture_contents_page}{Fermat\+: An Overture}. 
\end{DoxyParagraph}
\begin{DoxyParagraph}{}

\begin{DoxyItemize}
\item 1. \hyperlink{_overture_page}{Introduction} 
\begin{DoxyItemize}
\item 1.\+1. \hyperlink{_overture_page_MeshesSection}{The Mesh Geometry} 
\item 1.\+2. \hyperlink{_overture_page_CameraSection}{The Camera Model} 
\item 1.\+3. \hyperlink{_overture_page_BSDFSection}{The B\+S\+DF Model} 
\item 1.\+4. \hyperlink{_overture_page_LightSection}{The Light Source Model} 
\end{DoxyItemize}
\item 2. \hyperlink{_r_t_context_page}{Ray Tracing Contexts} 
\item 3. \hyperlink{_rendering_context_page}{The Rendering Context} 
\item 4. \hyperlink{_renderer_interface_page}{The Renderer Interface} 
\item 5. \hyperlink{_plugins_page}{Plugins} 
\item 6. \hyperlink{_hello_renderer_page}{Hello Renderer!} 
\begin{DoxyItemize}
\item 6.\+1. \hyperlink{_hello_renderer_page_HelloPTGeneratingPrimaryRaysSection}{Generating Primary Rays} 
\item 6.\+2. \hyperlink{_hello_renderer_page_HelloPTShadingVerticesSection}{Shading Vertices} 
\item 6.\+3. \hyperlink{_hello_renderer_page_HelloPTSolvingOcclusionSection}{Resolving Occlusion} 
\item 6.\+4. \hyperlink{_hello_renderer_page_HelloPTPluginSection}{The Plugin} 
\item 6.\+5. \hyperlink{_hello_renderer_page_HelloPTDoneSection}{We\textquotesingle{}re Done!} 
\end{DoxyItemize}
\item 7. \hyperlink{_p_t_lib_page}{P\+T\+Lib} 
\begin{DoxyItemize}
\item 7.\+1. \hyperlink{_p_t_lib_page_PTWavefrontSchedulingSection}{Wavefront Scheduling} 
\end{DoxyItemize}
\item 8. \hyperlink{_p_t_page}{The Path Tracer (Revisited)} 
\item 9. \hyperlink{_p_s_f_p_t_page}{The Path-\/\+Space Filtering Path Tracer} 
\item 10. \hyperlink{_b_p_t_lib_page}{B\+P\+T\+Lib} 
\begin{DoxyItemize}
\item 10.\+1. \hyperlink{_b_p_t_lib_page_BPTLibCoreSection}{B\+P\+T\+Lib\+Core Description} 
\item 10.\+2. \hyperlink{_b_p_t_lib_page_BPTLibSection}{B\+P\+T\+Lib Description} 
\item 10.\+3. \hyperlink{_b_p_t_lib_page_BPTExampleSection}{An Example} 
\end{DoxyItemize}
\item 11. \hyperlink{_modules_page}{Module List} 
\end{DoxyItemize}
\end{DoxyParagraph}
\hypertarget{index_FermatDependenciesSection}{}\section{Dependencies}\label{index_FermatDependenciesSection}
\begin{DoxyParagraph}{}
Fermat depends on the following external libraries\+: ~\newline

\begin{DoxyItemize}
\item \href{http://nvlabs.github.io/cub/}{\tt C\+UB} \+: contained in the package
\item a modification of Nathaniel Mc\+Clatchey\textquotesingle{}s \href{https://github.com/nmcclatchey/Priority-Deque/}{\tt priority\+\_\+deque} \+: contained in the package
\item \href{http://freeglut.sourceforge.net/}{\tt Free\+G\+L\+UT} \+: contained in the package
\item \href{http://assimp.org/}{\tt Assimp} \+: contained in the package
\item \href{https://developer.nvidia.com/cuda-80-ga2-download-archive}{\tt C\+U\+DA 10.\+0} \+: {\bfseries not contained} -\/ it should be separately downloaded and installed on the system
\item \href{https://developer.nvidia.com/optix}{\tt N\+V\+I\+D\+IA OptiX 6.\+0} \+: {\bfseries not contained} -\/ it should be separately downloaded and copied in the folder {\itshape contrib/\+OptiX}
\end{DoxyItemize}
\end{DoxyParagraph}
\begin{DoxyParagraph}{}
It also needs an N\+V\+I\+D\+IA Driver 418.\+81 or newer.
\end{DoxyParagraph}
\hypertarget{index_Licensing}{}\section{Licensing}\label{index_Licensing}
\begin{DoxyParagraph}{}
Fermat has been developed by \href{www.nvidia.com}{\tt N\+V\+I\+D\+IA Corporation} and is licensed under B\+SD.
\end{DoxyParagraph}
\hypertarget{index_Contributors}{}\section{Contributors}\label{index_Contributors}
\begin{DoxyParagraph}{}
\href{https://github.com/NVlabs/fermat}{\tt Fermat} is made and mantained by \href{jpantaleoni@nvidia.com}{\tt Jacopo Pantaleoni}.
\end{DoxyParagraph}
\hypertarget{index_NewsSection}{}\section{Versions and News}\label{index_NewsSection}
\begin{DoxyParagraph}{}
\tabulinesep=1mm
\begin{longtabu} spread 0pt [c]{*{2}{|X[-1]}|}
\hline
4/2/2019 ~\newline
 Fermat 2.\+0 &
\begin{DoxyItemize}
\item {\bfseries New Features\+:} ~\newline

\begin{DoxyItemize}
\item Added an OptiX 6.\+0 backend with full R\+TX support.
\item Added a new plugin renderer interface.
\item Added a path-\/space \hyperlink{struct_m_l_t}{M\+LT} renderer.
\item Added a novel \hyperlink{struct_direct_lighting_r_l}{reinforcement-\/learning based direct lighting sampler}.
\item Added an example plugin renderer.
\item Added a new scene description format and support for more formats with \href{http://assimp.org/}{\tt Assimp}.
\item Added a \href{https://www.pbrt.org/fileformat-v3.html}{\tt pbrt} scene importer.
\item Added comprehensive documentation.  
\end{DoxyItemize}
\end{DoxyItemize}\\\cline{1-2}
\end{longtabu}

\end{DoxyParagraph}
\begin{DoxyParagraph}{}
\tabulinesep=1mm
\begin{longtabu} spread 0pt [c]{*{2}{|X[-1]}|}
\hline
20/4/2018 ~\newline
 Fermat 1.\+0 &
\begin{DoxyItemize}
\item {\bfseries First Release}  
\end{DoxyItemize}\\\cline{1-2}
\end{longtabu}

\end{DoxyParagraph}
\hypertarget{index_DownloadSection}{}\section{Download}\label{index_DownloadSection}
\begin{DoxyParagraph}{}
You can download Fermat from Git\+Hub at\+: ~\newline
 \href{https://github.com/NVlabs/fermat}{\tt https\+://github.\+com/\+N\+Vlabs/fermat} ~\newline
 or directly clone the repository with the command\+: ~\newline
 git clone \href{https://github.com/NVlabs/fermat.io}{\tt https\+://github.\+com/\+N\+Vlabs/fermat.\+io}
\end{DoxyParagraph}
 